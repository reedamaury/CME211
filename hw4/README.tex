\documentclass{homework}

\title{Software Engineering for Scientists and Engineers (CME 211): Assignment 4}
\author{Amaury Reed}

\begin{document}

\maketitle

\section{Summary of Truss.py Program}
This program takes in two arguments of the directory where the folder containing the truss data is located. 
The folder should consists of the following data pertaining to the truss: 
\begin{enumerate}
\item the joint coordinates
\item beams and the corresponding joints 
\item support reaction forces and external forces corresponding to each joint coordinate. 
\end{enumerate}
The file with information corresponding to joint coordinate information should be named 
joints.dat and the file with the beam information should be named joints.dat. Any deviations from these file names 
will result in a runtime error. Additionally, if the directory in the command line argument does not exits, the 
program will output a runtime error. The program computes the by beam forces by summing the forces in the in x 
and y directions at each joint and constructing a coefficient matrix corresponding to the equations. The A matrix 
contains the coefficients of beam forces as well as the coefficients of the support reaction forces, while the B 
matrix contains coefficients of the external forces. It is important to note that every two rows of the dense A matrix corresponds to a 
joint - that is the sum of the forces in the x and y directions for the joint. However, a sparse matrix was used in this program for efficiency of memory allocation. The linear system of equations is sovled using the follwing formula. 

\begin{equation}
    Ax = B
\end{equation}

If the coefficient matrix A is under- or over-defined the program will return a 
runtime error. If an output directory argument is given, the funciton will save a plot of the truss geometry to 
the said directory.

\end{document}
